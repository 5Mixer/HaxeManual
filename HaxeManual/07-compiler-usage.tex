\chapter{Compiler Usage}
\label{compiler-usage}

\paragraph{Basic Usage}

The Haxe Compiler is typically invoked from command line with several arguments which have to answer two questions:

\begin{itemize}
	\item What should be compiled?
	\item What should the output be?
\end{itemize}
	
To answer the first question, it is usually sufficient to provide a class path via the \ic{-cp path} argument, along with the main class to be compiled via the \ic{-main dot_path} argument. The Haxe Compiler then resolves the main class file and begins compilation.

The second question usually comes down to providing an argument specifying the desired target. Each Haxe target has a dedicated command line switch, such as \ic{-js file_name} for Javascript and \ic{-php directory} for PHP. Depending on the nature of the target, the argument value is either a file name (for \ic{-js}, \ic{-swf} and \ic{neko}) or a directory path.

\paragraph{Common arguments}

\emph{Input:}

\begin{description}
	\item[\ic{-cp path}] Adds a class path where \ic{.hx} source files or packages (sub-directories) can be found.
	\item[\ic{-lib library_name}] Adds a \Fullref{haxelib} library.
	\item[\ic{-main dot_path}] Sets the main class.
\end{description}

\emph{Output:}

\begin{description}
	\item[\ic{-js file_name}] Generates \tref{Javascript}{target-javascript} source code in specified file.
	\item[\ic{-as3 directory}] Generates Actionscript 3 source code in specified directory.
	\item[\ic{-swf file_name}] Generates the specified file as \tref{Flash}{target-flash} .swf.
	\item[\ic{-neko file_name}] Generates \tref{Neko}{target-neko} binary as specified file.
	\item[\ic{-php directory}] Generates \tref{PHP}{target-php} source code in specified directory.
	\item[\ic{-cpp directory}] Generates \tref{C++}{target-cpp} source code in specified directory and compiles it using native C++ compiler.
	\item[\ic{-cs directory}] Generates \tref{C\#}{target-cs} source code in specified directory.
	\item[\ic{-java directory}] Generates \tref{Java}{target-java} source code in specified directory and compiles it using the Java Compiler.
	\item[\ic{-python file_name}] Generates \tref{Python}{target-python} source code in the specified file.
\end{description}

