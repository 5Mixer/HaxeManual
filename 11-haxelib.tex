\chapter{Haxelib}
\label{haxelib}

Haxelib is the library manager that comes with any Haxe distribution. Connected to a central repository, it allows submitting and retrieving libraries and has multiple features beyond that. Available libraries can be found at \url{http://lib.haxe.org}.

A basic Haxe library is a collection of \ic{.hx} files. That is, libraries are distributed by source code by default, making it easy to inspect and modify their behavior. Each library is identified by a unique name, which is utilized when telling the Haxe Compiler which libraries to use for a given compilation.

\section{Using a Haxe library with the Haxe Compiler}
\label{haxelib-using-haxe}

Any installed Haxe library can be made available to the compiler through the \ic{-lib <library-name>} argument. This is very similiar to the \ic{-cp <path>} argument, but expects a library name instead of a directory path. These commands are explained thoroughly in \Fullref{compiler-usage}.

For our exemplary usage we chose a very simple Haxe library called ``random''. It provides a set of static convenience methods to achieve various random effects, such as picking a random element from an array.

\haxe{assets/HaxelibRandom.hx}

Compiling this without any \ic{-lib} argument causes an error message along the lines of \ic{Unknown identifier : Random}. This shows that installed Haxe libraries are not available to the compiler by default unless they are explicitly added. A working command line for above program is \ic{haxe -lib random -main Main --interp}.

If the compiler emits an error \ic{Error: Library random is not installed : run 'haxelib install random'} the library has to be installed via the \ic{haxelib} command first. As the error message suggests, this is achieved through \ic{haxelib install random}. We will learn more about the \ic{haxelib} command in \Fullref{haxelib-using}.



\section{haxelib.json}
\label{haxelib-json}

Each Haxe library requires a \ic{haxelib.json} file in which the following attributes are defined:

\begin{description}
	\item[name:] The name of the library. It must contain at least 3 characters among the following: \ic{\[A-Za-z0-9_-.\]}. In particular, no spaces are allowed.
	\item[url:] The URL of the library, i.e. where more information can be found.
	\item[license:] The license under which the library is released. Can be \ic{GPL}, \ic{LGPL}, \ic{BSD}, \ic{Public} (for Public Domain) or \ic{MIT}.
	\item[tags:] An array of tag-strings which are used on the repository website to sort libraries.
	\item[description:] The description of what the library is doing.
	\item[version:] The version string of the library. This is detailed in \Fullref{haxelib-json-versioning}.
	\item[classPath:] The path string to the source files.
	\item[releasenote:] The release notes of the current version.
	\item[contributors:] An array of user names which identify contributors to the library. 
	\item[dependencies:] An object describing the dependencies of the library. This is detailed in \Fullref{haxelib-json-dependencies}.
\end{description}

The following JSON is a simple example of a haxelib.json:

\begin{lstlisting}
{
  "name": "useless_lib",
  "url" : "https://github.com/jasononeil/useless/",
  "license": "MIT",
  "tags": ["cross", "useless"],
  "description": "This library is useless in the same way on every platform.",
  "version": "1.0.0",
  "releasenote": "Initial release, everything is working correctly.",
  "contributors": ["Juraj","Jason","Nicolas"],
  "dependencies": {
    "tink_macro": "",
    "nme": "3.5.5"
  }
}
\end{lstlisting}

\subsection{Versioning}
\label{haxelib-json-versioning}

Haxelib uses a simplified version of \href{http://semver.org/}{SemVer}. The basic format is this:

\begin{lstlisting}
major.minor.patch
\end{lstlisting}

These are the basic rules:

\begin{itemize}
	\item Major versions are incremented when you break backwards compatibility - so old code will not work with the new version of the library.
	\item Minor versions are incremented when new features are added.
	\item Patch versions are for small fixes that do not change the public API, so no existing code should break.
	\item When a minor version increments, the patch number is reset to 0. When a major version increments, both the minor and patch are reset to 0.
\end{itemize}

Examples:

\begin{description}
\item["0.0.1":] A first release.  Anything with a "0" for the major version is subject to change in the next release - no promises about API stability!
\item["0.1.0":] Added a new feature!   Increment the minor version, reset the patch version
\item["0.1.1":] Realised the new feature was broken.  Fixed it now, so increment the patch version
\item["1.0.0":] New major version, so increment the major version, reset the minor and patch versions.   You promise your users not to break this API until you bump to 2.0.0
\item["1.0.1":] A minor fix
\item["1.1.0":] A new feature
\item["1.2.0":] Another new feature
\item["2.0.0":] A new version, which might break compatibility with 1.0.  Users are to upgrade cautiously.
\end{description}

If this release is a preview (Alpha, Beta or Release Candidate), you can also include that, with an optional release number:

\begin{lstlisting}
major.minor.patch-(alpha/beta/rc).release
\end{lstlisting}

Examples:

\begin{description}
\item["1.0.0-alpha":] The alpha of 1.0.0 - use with care, things are changing!
\item["1.0.0-alpha.2":] The 2nd alpha
\item["1.0.0-beta":] Beta - things are settling down, but still subject to change.
\item["1.0.0-rc.1":] The 1st release candidate for 1.0.0 - you shouldn't be adding any more features now
\item["1.0.0-rc.2":] The 2nd release candidate for 1.0.0
\item["1.0.0":] The final release!  
\end{description}


\subsection{Dependencies}
\label{haxelib-json-dependencies}

As of Haxe 3.1.0, haxelib supports only exact version matching for dependencies. Dependencies are defined as part of the \tref{haxelib.json}{haxelib-json}, with the library name serving as key and the expected version (if required) as value in the format described in \Fullref{haxelib-json-versioning}.

We have seen an example of this when introducing haxelib.json:

\begin{lstlisting}
"dependencies": {
  "tink_macros": "",
  "nme": "3.5.5"
}
\end{lstlisting}

This adds two dependencies to the given Haxe library:

\begin{enumerate}
	\item The library ``tink_macros'' can be used in any version. Haxelib will then always try to use the latest version.
	\item The library ``nme'' is required in version ``3.5.5''. Haxelib will make sure that this exact version is used, avoiding potential breaking changes with future versions.
\end{enumerate}


\section{extraParams.hxml}
\label{haxelib-extraParams}

If you add a file named \ic{extraParams.hxml} to your library root (at the same level as \ic{haxelib.json}), these parameters will be automatically added to the compilation parameters when someone use your library with \ic{-lib}.


\section{Using Haxelib}
\label{haxelib-using}

If the \ic{haxelib} command is executed without any arguments, it prints an exhaustive list of all available arguments. Access the \url{http://lib.haxe.org} website to view all the libraries available. 

The following commands are available:

\begin{description}
	\item[Basic]
		\begin{description}
			\item[\ic{haxelib install [project-name] [version]}] installs the given project. You can optionally specify a specific version to be installed. By default, latest released version will be installed.
			\item[\ic{haxelib update [project-name]}] updates a single library to their latest version. 
			\item[\ic{haxelib upgrade}] upgrades all the installed projects to their latest version. This command prompts a confirmation for each upgradeable project.
			\item[\ic{haxelib remove project-name [version]}] removes complete project or only a specified version if specified.
			\item[\ic{haxelib list}] lists all the installed projects and their versions. For each project, the version surrounded by brackets is the current one.
			\item[\ic{haxelib set [project-name] [version]}] changes the current version for a given project. The version must be already installed.
		\end{description}
		
	\item[Information]
		\begin{description}
			\item[\ic{haxelib search [word]}] lists the projects which have either a name or description matching specified word.
			\item[\ic{haxelib info [project-name]}] gives you information about a given project.
			\item[\ic{haxelib user [user-name]}] lists information on a given Haxelib user.
			\item[\ic{haxelib config}] prints the Haxelib repository path. This is where Haxelib get installed by default.
			\item[\ic{haxelib path [project-name]}] prints paths to libraries and its dependencies (defined in \ic{haxelib.xml}).
		\end{description}
		
	\item[Development]
		\begin{description}
			\item[\ic{haxelib submit [project.zip]}] submits a package to Haxelib. If the user name is unknown, you'll be first asked to register an account. If the user already exists, you will be prompted for your password. If the project does not exist yet, it will be created, but no version will be added. You will have to submit it a second time to add the first released version. If you want to modify the project url or description, simply modify your \ic{haxelib.xml} (keeping version information unchanged) and submit it again.
			\item[\ic{haxelib register [project-name]}] submits or update a library package.
			\item[\ic{haxelib local [project-name]}] tests the library package. Make sure everything (both installation and usage) is working correctly before submitting, since once submitted, a given version cannot be updated.
			\item[\ic{haxelib dev [project-name] [directory]}] sets a development directory for the given project. To set project directory back to global location, run command and omit directory.
			\item[\ic{haxelib git [project-name] [git-clone-path] [branch] [subdirectory]}] uses git repository as library. This is useful for using a more up-to-date development version, a fork of the original project, or for having a private library that you do not wish to post to Haxelib. When you use \ic{haxelib upgrade} any libraries that are installed using GIT will automatically pull the latest version.
		\end{description}
		
	\item[Miscellaneous]
		\begin{description}
			\item[\ic{haxelib setup}] sets the Haxelib repository path. To print current path use \ic{haxelib config}.
			\item[\ic{haxelib selfupdate}] updates Haxelib itself. It will ask to run \ic{haxe update.hxml} after this update.
			\item[\ic{haxelib convertxml}] converts \ic{haxelib.xml} file to \ic{haxelib.json}.
			\item[\ic{haxelib run [project-name] [parameters]}] runs the specified library with parameters. Requires  a precompiled Haxe/Neko \ic{run.n} file in the library package. This is useful if you want users to be able to do some post-install script that will configure some additional things on the system. Be careful to trust the project you are running since the script can damage your system.
			\item[\ic{haxelib proxy}] setup the Http proxy.
		\end{description}
\end{description}
