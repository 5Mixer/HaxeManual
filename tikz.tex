\usepackage{tikz}
\usepackage{tkz-euclide}
\usetikzlibrary{shapes.geometric, arrows}
\usepackage{tikz-qtree}


\tikzstyle{flowchartitem} = [
	minimum width = 2cm,
	minimum height = 1cm,
	text centered,
	draw = black,
]

\tikzstyle{startstop} = [
	flowchartitem,
	rectangle,
	rounded corners,
	fill = red!20
]
 
\tikzstyle{io} = [
	flowchartitem,
	trapezium,
	trapezium left angle = 70,
	trapezium right angle = 110,
	fill = blue!20
]

\tikzstyle{process} = [
	flowchartitem,
	rectangle,
	fill = orange!20
]
\tikzstyle{decision} = [
	flowchartitem,
	diamond,
	fill = green!20
]

\tikzstyle{valueNone} = [ fill = red!50 ]
\tikzstyle{valueSome} = [ fill = yellow!50 ]
\tikzstyle{valueAll} = [ fill = green!50 ]

\newenvironment*{tikzfigure}[1][]{
	\begin{figure}[h!]
	\begin{tikzfigureinternal}[#1]
}{
	\end{tikzfigureinternal}
	\end{figure}
}

\newenvironment*{tikzsubfigure}[1][]{
	\begin{subfigure}[b]{0.45\textwidth}
	\begin{tikzfigureinternal}[#1]
}{
	\end{tikzfigureinternal}
	\end{subfigure}
}

\newenvironment*{tikzfigureinternal}[1][]{
	\centering
	\gdef\tcapt{#1}
	\begin{tikzpicture}[scale=1]
}{
	\end{tikzpicture}
	\caption{\tcapt}
}

\tikzstyle{flowchartArrow} = [
	thick,
	->,
	>=stealth,
	decoration = {
		markings,
		mark = at position 1 with {
			\arrow[scale=2]{>}
		}
	},
	postaction = {decorate}
]

\newenvironment*{flowchart}[2]{
	\label{#1}
	\begin{tikzfigure}[#2]
	\tikzset {
		level distance = 3cm,
		grow = down,
		edge from parent/.append style = { flowchartArrow }
	}
}{
	\end{tikzfigure}
}
