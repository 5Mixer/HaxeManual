\chapter{Standard Library}
\label{std}
\state{NoContent}

Standard library

\section{String}
\label{std-String}

\define[Type]{String}{define-string}{A String is a sequence of characters.}

%TODO: utf8 crap %

\section{Data Structures}
\label{std-ds}
\state{NoContent}

\subsection{Array}
\label{std-Array}

An \type{Array} is a \emph{collection} for storing elements.  It has one \tref{type parameter}{type-system-type-parameters} and all elements of the array must be of the specified type.  Alternatively, arrays of mixed types are allowed if the type parameter is \Fullref{types-dynamic}.  See the below code snippet for an example. 
\trivia{Dynamic Arrays}{In Haxe 2, mixed type array declarations were allowed.  In Haxe 3, arrays can have mixed types ONLY if they are explicitly declared as \emph{\expr{Array<Dynamic>}}.}
The following example shows some basic examples of working with arrays:
\haxe{assets/ArrayExample.hx}

\todo{make sure this is true about static targets}
In Haxe, arrays are unbounded:  accessing or assigning an index outside the current size of the array does NOT result in an exception.  When assigning, the array grows and the inbetween elements are assigned \emph{null} (or the \emph{static target} alternative: see \Fullref{types-nullability} for more details).  When accessing, \emph{null} is returned.  Accessing a negative index in an array also returns \emph{null}.

Arrays can be \emph{iterated over} using a \Fullref{expression-for} loop.  However removing elements while iterating over an array is error prone (but just fine with a \emph{List}).

\todo{Maybe we should introduce \textbackslash api\{type name\}}
See \Fullref{expression-array-declaration} for array initialization.  New arrays can also be created by \Fullref{lf-array-comprehension}.  The \href{http://api.haxe.org/Array.html}{Array API} has details about Array methods.

\subsection{Vector}
\label{std-vector}

A \type{Vector} is an optimized fixed-length \emph{collection} of elements. Much like \tref{Array}{std-Array}, it has one \tref{type parameter}{type-system-type-parameters} and all elements of a vector must be of the specified type, it can be \emph{iterated over} using a \tref{for loop}{expression-for} and accessed using \tref{array access syntax}{types-abstract-array-access}. However, unlike \type{Array} and \type{List}, vector length is specified on creation and cannot be changed later.

\haxe{assets/Vector.hx}

\type{haxe.ds.Vector} is implemented as an abstract type (\ref{types-abstract}) over a native array implementation for given target and can be faster for fixed-size collections, because the memory for storing its elements is pre-allocated.

\subsection{List}
\label{std-List}
A \type{List} is a \emph{collection} for storing elements.  On the surface, a list is similar to an \Fullref{std-Array}.  However, the underlying implementation is very different.  This results in several functional differences:
\todo{I hope none of these are lies -C }
\begin{enumerate}
	\item A list can not be indexed using square brackets, i.e. \expr{[0]}.
	\item A list can not be initialized.
	\item There are no list comprehensions.
	\item A list can freely modify/add/remove elements while iterating over them.
\end{enumerate}

See the \href{http://api.haxe.org/List.html}{List API} for details about the list methods.  A simple example for working with lists:
\haxe{assets/ListExample.hx}

\subsection{GenericStack}
\label{std-GenericStack}
A \type{GenericStack}, like \type{Array} and \type{List} is a container for storing elements.  It has one \tref{type parameter}{type-system-type-parameters} and all elements of the array must be of the specified type. See the \href{http://api.haxe.org/haxe/ds/GenericStack.html}{GenericStack API} for details about its methods.  Here is a small example program for initializing and working with a \type{GenericStack}.
\haxe{assets/GenericStackExample.hx}
\trivia{FastList}{In Haxe 2, the GenericStack class was known as FastList.  Since its behavior more closely resembled a typical stack, the name was changed for Haxe 3.}
The \emph{Generic} in \type{GenericStack} is literal.  It is attributed with the \expr{:generic} metadata.  Depending on the target, this can lead to improved performance on static targets.  See \Fullref{type-system-generic} for more details.
\subsection{Map}
\label{std-Map}

A \type{Map} is a container composed of \emph{key}, \emph{value} pairs.  A \type{Map} is also commonly referred to as an associative array, dictionary, or symbol table.  The following code gives a short example of working with maps:

\haxe{assets/MapExample.hx}

See the \href{http://api.haxe.org/Map.html}{Map API} for details of its methods.

Under the hood, a \type{Map} is an \Fullref{types-abstract} type.  At compile time, it gets converted to one of several specialized types depending on the \emph{key} type:
\begin{itemize}
	\item haxe.ds.StringMap
	\item haxe.ds.IntMap
	\item haxe.ds.EnumValueMap
	\item haxe.ds.ObjectMap
\end{itemize}
So, at runtime, the \type{Map} type does not exist, and has been replaced with one of the above objects.  

\subsection{Option}
\label{std-Option}

\section{Regular Expressions}
\label{std-regex}

Haxe has built-in support for \emph{regular expressions}\footnote{http://en.wikipedia.org/wiki/Regular_expression}. They can be used to verify the format of a string, transform a string or extract some regular data from a given text.

Haxe has special syntax for creating regular expressions. We can create a regular expression object by typing it between the \expr{\textasciitilde/} combination and a single \expr{/} character:

\begin{lstlisting}
var r = ~/haxe/i;
\end{lstlisting}

Alternatively, we can create regular expression with regular syntax:

\begin{lstlisting}
var r = new EReg("haxe", "i");
\end{lstlisting}

First argument is a string with regular expression pattern, second one is a string with \emph{flags} (see below).

We can use standard regular expression patterns such as:
\begin{itemize}
	\item \expr{.} any character
	\item \expr{*} repeat zero-or-more
	\item \expr{+} repeat one-or-more
	\item \expr{?} optional zero-or-one
	\item \expr{[A-Z0-9]} character ranges
	\item \expr{[\textasciicircum\textbackslash r\textbackslash n\textbackslash t]} character not-in-range
	\item \expr{(...)} parenthesis to match groups of characters
	\item \expr{\textasciicircum} beginning of the string (beginning of a line in multiline matching mode)
	\item \expr{\$} end of the string (end of a line in multiline matching mode)
	\item \expr{|} "OR" statement.
\end{itemize}

For example, the following regular expression matches valid email addresses:
\begin{lstlisting}
~/[A-Z0-9._\%-]+@[A-Z0-9.-]+\.[A-Z][A-Z][A-Z]?/i;
\end{lstlisting}

Please notice that the \expr{i} at the end of the regular expression is a \emph{flag} that enables case-insensitive matching.

The possible flags are the following:
\begin{itemize}
	\item \expr{i} case insensitive matching
	\item \expr{g} global replace or split, see below
	\item \expr{m} multiline matching, \expr{\textasciicircum} and \expr{\$} represent the beginning and end of a line
	\item \expr{s} the dot \expr{.} will also match newlines \emph{(Neko, C++, PHP and Java targets only)}
	\item \expr{u} use UTF-8 matching \emph{(Neko and C++ targets only)}
\end{itemize}

\subsection{Matching}
\label{std-regex-match}

Probably one of the most common uses for regular expressions is checking whether a string matches the specific pattern. The \expr{match} method of a regular expression object can be used to do that:
\haxe{assets/ERegMatch.hx}

\subsection{Groups}
\label{std-regex-groups}

Specific information can be extracted from a matched string by using \emph{groups}. If \expr{match()} returns true, we can get groups using the \expr{matched(X)} method, where X is the number of a group defined by regular expression pattern:

\haxe{assets/ERegGroups.hx}

Note that group numbers start with 1 and \expr{r.matched(0)} will always return the whole matched substring.

The \expr{r.matchedPos()} will return the position of this substring in the original string:

\haxe{assets/ERegMatchPos.hx}

Additionally, \expr{r.matchedLeft()} and \expr{r.matchedRight()} can be used to get substrings to the left and to the right of the matched substring:

\haxe{assets/ERegMatchLeftRight.hx}

\subsection{Replace}
\label{std-regex-replace}

A regular expression can also be used to replace a part of the string:

\haxe{assets/ERegReplace.hx}

We can use \expr{\$X} to reuse a matched group in the replacement:

\haxe{assets/ERegReplaceGroups.hx}

\subsection{Split}
\label{std-regex-split}

A regular expression can also be used to split a string into several substrings:

\haxe{assets/ERegSplit.hx}

\subsection{Map}
\label{std-regex-map}

The \expr{map} method of a regular expression object can be used to replace matched substrings using a custom function:

\haxe{assets/ERegMap.hx}

This function takes a regular expression object as its first argument so we may use it to get additional information about the match being done.

\subsection{Implementation Details}
\label{std-regex-implementation-details}

Regular Expressions are implemented:
\begin{itemize}
	\item in JavaScript, the runtime is providing the implementation with the object RegExp.
	\item in Neko and C++, the PCRE library is used
	\item in Flash, PHP, C\# and Java, native implementations are used
	\item in Flash 6/8, the implementation is not available
\end{itemize}


\section{Math}
\label{std-math}

Haxe includes a floating point math library for some common mathematical operations.  Most of the fuctions operate on and return \type{floats}.  However, an \type{Int} can be used where a \type{Float} is expected, and Haxe also converts \type{Int} to \type{Float} during most numeric operations  (see \Fullref{types-numeric-operators} for more details).

Here are some example uses of the math library.  See the \href{http://api.haxe.org/Math.html}{Math API} for all available functions.

\haxe{assets/MathExample.hx}

\subsection{Special Numbers}
\label{std-math-special-numbers}
The math library has definitions for several special numbers:
\begin{itemize}
	\item NaN (Not a Number): returned when a mathmatically incorrect operation is executed, e.g. Math.sqrt(-1)
	\item POSITIVE_INFINITY: e.g. divide a positive number by zero
	\item NEGATIVE_INFINITY: e.g. divide a negative number by zero
	\item PI : 3.1415...
\end{itemize}

\subsection{Mathematical Errors}
\label{std-math-mathematical-errors}
Although neko can fluidly handle mathematical errors, like division by zero, this is not true for all target.  Depending on the target, mathematical errors may produce exceptions and ultimately errors.

\subsection{Integer Math}
\label{std-math-integer-math}
If you are targeting a platform that can utilize integer operations, e.g. integer division, it should be wrapped in \emph{Std.int()} for improved performance.  The Haxe Compiler can then optimize for integer operations.  An example:
\begin{lstlisting}
	var intDivision = Std.int(6.2/4.7);
\end{lstlisting}
\todo{I think C++ can use integer operatins, but I don't know about any other targets. Only saw this mentioned in an old discussion thread, still true?}

\subsection{Extensions}
\label{std-math-extensions}
It is common to see \Fullref{lf-static-extension} used with the math library.  This code shows a simple example:  
\haxe{assets/MathStaticExtension.hx}
\haxe{assets/MathExtensionUsage.hx}


\section{Lambda}
\label{std-Lambda}

\section{Reflection}
\label{std-reflection}

\section{Serialization}
\label{std-serialization}

\section{Json}
\label{std-Json}

Haxe provides built-in support for (de-)serializing \emph{JSON}\footnote{http://en.wikipedia.org/wiki/JSON} data via the \type{haxe.Json} class.

\subsection{Parsing JSON}
\label{std-Json-parsing}

Use the \expr{haxe.Json.parse} static method to parse \emph{JSON} data and obtain a Haxe value from it:
\haxe{assets/JsonParse.hx}

Note that the type of the object returned by \expr{haxe.Json.parse} is \expr{Dynamic}, so if the structure of our data is well-known, we may want to specify a type using \tref{anonymous structures}{types-anonymous-structure}. This way we provide compile-time checks for accessing our data and most likely more optimal code generation, because compiler knows about types in a structure:
\haxe{assets/JsonParseTyped.hx}

\subsection{Encoding JSON}
\label{std-Json-encoding}

Use the \expr{haxe.Json.stringify} static method to encode a Haxe value into a \emph{JSON} string:
\haxe{assets/JsonStringify.hx}

\subsection{Implementation details}
\label{std-Json-implementation-details}

The \type{haxe.Json} API automatically uses native implementation on targets where it is available, i.e. \emph{JavaScript}, \emph{Flash} and \emph{PHP} and provides its own implementation for other targets.

Usage of Haxe own implementation can be forced with \expr{-D haxeJSON} compiler argument. This will also provide serialization of \tref{enums}{types-enum-instance} by their index, \tref{maps}{std-Map} with string keys and class instances.

Older browsers (Internet Explorer 7, for instance) may not have native \emph{JSON} implementation. In case it's required to support them, we can include one of the JSON implementations available on the internet in the HTML page. Alternatively, a \expr{-D old_browser} compiler argument that will make \type{haxe.Json} try to use native JSON and fallback to its own implementation in case it's not available can be used.

\section{Xml}
\label{std-Xml}

\section{Input/Output}
\label{std-input-output}

\section{Sys/sys}
\label{std-sys}
