\chapter{Compiler Features}
\label{cr-features}
\state{NoContent}

\section{Dead Code Elimination}
\label{cr-dce}

Dead Code Elimination, or \emph{DCE}, is a compiler feature which removes unused code from the output. After typing, the compiler evaluates the DCE entry-points (usually the main-method) and recursively determines which fields and types are used. Used fields are marked accordingly and unmarked fields are then removed from their classes.

DCE has three modes which are set when invoking the command line:

\begin{description}
	\item[-dce std:] Only classes in the Haxe Standard Library are affected by DCE. This is the default setting on all targets but \target{Javascript}.
	\item[-dce no:] No DCE is performed.
	\item[-dce full:] All classes are affected by DCE. This is the default setting when targeting \target{Javascript}.
\end{description}
The DCE-algorithm works well with typed code, but may fail when \tref{dynamic}{types-dynamic} or \tref{reflection}{std-reflection} is involved. This may require explicit marking of fields or classes as being used by attributing the following metadata:

\begin{description}
	\item[\expr{@:keep}:] If used on a class, the class along with all fields is unaffected by DCE. If used on a field, that field is unaffected by DCE.
	\item[\expr{@:keepSub}:] If used on a class, it works like \expr{@:keep} on the annotated class as well as all subclasses.
	\item[\expr{@:keepInit}:] Usually, a class which had all fields removed by DCE (or is empty to begin with) is removed from the output. By using this metadata, empty classes are kept.
\end{description}

If a class needs to be marked with \expr{@:keep} from the command line instead of editing its source code, there is a compiler macro available for doing so: \expr{--macro keep('type dot path')} See the \href{http://api.haxe.org/haxe/macro/Compiler.html#keep}{haxe.macro.Compiler.keep API} for details of this macro. It will mark package, module or sub-type to be kept by DCE and includes them for compilation.
 
The compiler automatically defines the flag \expr{dce} with a value of either \expr{"std"}, \expr{"no"} or \expr{"full"} depending on the active mode. This can be used in \tref{conditional compilation}{lf-condition-compilation}.

\trivia{DCE-rewrite}{DCE was originally implemented in Haxe 2.07. This implementation considered a function to be used when it was explicitly typed. The problem with that was that several features, most importantly interfaces, would cause all class fields to be typed in order to verify type-safety. This effectively subverted DCE completely, prompting the rewrite for Haxe 2.10.}

\trivia{DCE and try.haxe.org}{DCE for the \type{Javascript} target saw vast improvements when the website \url{http://try.haxe.org} was published. Initial reception of the generated \target{Javascript} code was mixed, leading to a more fine-grained selection of which code to eliminate.}

\section{Completion}
\label{cr-completion}

\subsection{Field Access}
\label{Field access}

\subsection{Call Arguments}
\label{Call arguments}

\subsection{Usage}
\label{usage}

\subsection{Position}
\label{position}

\subsection{Metadata}
\label{metadata}

\section{Resources}
\label{cr-resources}
\flag{fold}{true}

Haxe provides simple resource embedding system that can be used for embedding  files directly into the compiled application.

While it may be not optimal to embed large assets, like images or music in the application file, it comes in very handy to embed smaller resources, like configuration or XML data.

\subsection{Embedding resources}
\label{cr-resources-embed}

External files are embedded using the \emph{-resource} compiler argument:

\todo{what to use for listing of non-haxe code like hxml?}
\begin{lstlisting}
-resource hello_message.txt@welcome
\end{lstlisting}

The string after the \emph{@} symbol is the \emph{resource identifier}. It will used in the code for retrieving the resource. It can be omitted (together with the \emph{@} symbol), then the file name will become a resource identifier.

\subsection{Retrieving text resources}
\label{cr-resources-getString}

To retrieve the content of an embedded resource, we use the static method \emph{getString} of \type{haxe.Resource} passing a \emph{resource identifier} to it:

\haxe{assets/ResourceGetString.hx}

The code above will display the content of the \emph{hello_message.txt} file that we included earlier using \emph{welcome} as the identifier.

\subsection{Retrieving binary resources}
\label{cr-resources-getBytes}

While it's not recommended to embed large binary files in the application, it still may be useful to embed binary data. Binary representation of an embedded resource can be accessed using the static method \emph{getBytes} of \type{haxe.Resource}:

\haxe{assets/ResourceGetBytes.hx}

The return type of \emph{getBytes} method is \type{haxe.io.Bytes}, an object providing access to individual bytes of the data.

\subsection{Implementation details}
\label{cr-resources-impl}

Haxe uses target platform's native resource embedding, if there is one, providing its own implementation otherwise.

\begin{itemize}
\item \emph{Flash} resources are embedded as ByteArray definitions
\item \emph{C\#} resources are included in the compiled assembly
\item \emph{Java} resources are packed in the resulting JAR file
\item \emph{C++} resources are stored in global byte array constants.
\item \emph{JavaScript} resources are serialized in Haxe serialization format and stored in a static field of \type{haxe.Resource} class.
\item \emph{Neko} resources are stored as strings in a static field of \type{haxe.Resource} class.
\end{itemize}